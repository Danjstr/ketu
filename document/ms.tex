%
%  RULES OF THE GAME
%
%  * 80 characters
%  * line breaks at the ends of sentences
%  * eqnarrys ONLY
%  * that is all.
%

\documentclass[12pt,preprint]{aastex}

\pdfoutput=1

\include{vc}

\usepackage{color,hyperref}
\definecolor{linkcolor}{rgb}{0,0,0.5}
\hypersetup{colorlinks=true,linkcolor=linkcolor,citecolor=linkcolor,
            filecolor=linkcolor,urlcolor=linkcolor}
\usepackage{url}
\usepackage{amssymb,amsmath}
\usepackage{subfigure}

\newcommand{\project}[1]{\emph{#1}}
\newcommand{\kepler}{\project{Kepler}}
\newcommand{\terra}{\project{TERRA}}
\newcommand{\license}{MIT License}

\newcommand{\paper}{\textsl{Article}}

\newcommand{\foreign}[1]{\emph{#1}}
\newcommand{\etal}{\foreign{et\,al.}}
\newcommand{\etc}{\foreign{etc.}}
\newcommand{\True}{\foreign{True}}
\newcommand{\Truth}{\foreign{Truth}}

\newcommand{\figref}[1]{\ref{fig:#1}}
\newcommand{\Fig}[1]{Figure~\figref{#1}}
\newcommand{\fig}[1]{\Fig{#1}}
\newcommand{\figlabel}[1]{\label{fig:#1}}
\newcommand{\Tab}[1]{Table~\ref{tab:#1}}
\newcommand{\tab}[1]{\Tab{#1}}
\newcommand{\tablabel}[1]{\label{tab:#1}}
\newcommand{\Eq}[1]{Equation~(\ref{eq:#1})}
\newcommand{\eq}[1]{\Eq{#1}}
\newcommand{\eqalt}[1]{Equation~\ref{eq:#1}}
\newcommand{\eqlabel}[1]{\label{eq:#1}}
\newcommand{\Sect}[1]{Section~\ref{sect:#1}}
\newcommand{\sect}[1]{\Sect{#1}}
\newcommand{\sectalt}[1]{\ref{sect:#1}}
\newcommand{\App}[1]{Appendix~\ref{sect:#1}}
\newcommand{\app}[1]{\App{#1}}
\newcommand{\sectlabel}[1]{\label{sect:#1}}

\newcommand{\dd}{\ensuremath{\,\mathrm{d}}}
\newcommand{\bvec}[1]{\ensuremath{\boldsymbol{#1}}}
\newcommand{\appropto}{\mathrel{\vcenter{
  \offinterlineskip\halign{\hfil$##$\cr
    \propto\cr\noalign{\kern2pt}\sim\cr\noalign{\kern-2pt}}}}}
\newcommand{\densityunit}{{\ensuremath{\mathrm{nat}^{-2}}}}

% TO DOS
\newcommand{\todo}[3]{{\color{#2} \emph{#1} TODO: #3}}
\newcommand{\dfmtodo}[1]{\todo{DFM}{red}{#1}}
\newcommand{\hoggtodo}[1]{\todo{HOGG}{blue}{#1}}

\begin{document}

\title{%
    Transiting exoplanet search
}

\newcommand{\nyu}{2}
\newcommand{\mpia}{3}
\newcommand{\cds}{4}
\newcommand{\mpis}{5}
\author{%
    Daniel~Foreman-Mackey\altaffilmark{1,\nyu},
    David~W.~Hogg\altaffilmark{\nyu,\mpia,\cds},
    Bernhard Sch\"olkopf\altaffilmark{\mpis},
    \etal
}
\altaffiltext{1}         {To whom correspondence should be addressed:
                          \url{danfm@nyu.edu}}
\altaffiltext{\nyu}      {Center for Cosmology and Particle Physics,
                          Department of Physics, New York University,
                          4 Washington Place, New York, NY, 10003, USA}
\altaffiltext{\mpia}     {Max-Planck-Institut f\"ur Astronomie,
                          K\"onigstuhl 17, D-69117 Heidelberg, Germany}
\altaffiltext{\cds}      {Center for Data Science,
                          New York University,
                          4 Washington Place, New York, NY, 10003, USA}
\altaffiltext{\mpis}     {Max Planck Institute for Intelligent Systems
                          Spemannstrasse 38, 72076 T\"ubingen, Germany}

\begin{abstract}

Let's find us some exoplanets.

\end{abstract}

\keywords{%
methods: data analysis
---
methods: statistical
---
catalogs
---
planetary systems
---
stars: statistics
}

\section{Introduction}

\section{The noise model}

The \kepler\ light curves are full of systematic effects, both astrophysical
and instrumental.
The standard method for dealing with these effects is to filter the data to
remove these artifacts---presearch data conditioning (PDC), ARC, \etc\
\dfmtodo{CITE}.
This procedure will always \emph{overfit} and reduce the strength of the
signals of interest.
To deal with this, we simultaneously model the transit and the noise.

There are a lot of physical effects involved in generating the noise and it
would be computationally intractable to directly model them all.
Instead, we use a flexible Gaussian process as an effective model.
In this model, the likelihood of the data is given by
\begin{eqnarray}
\ln p(\bvec{y}\,|\,\bvec{\theta},\,\bvec{\alpha},\,\bvec{x}) &=&
-\frac{1}{2} \left[ \bvec{y} - f_{\bvec{\theta}}(\bvec{x}) \right]^\mathrm{T}
\, K_{\bvec{\alpha}}^{-1}(\bvec{x}) \,
\left[ \bvec{y} - f_{\bvec{\theta}}(\bvec{x}) \right]
-\frac{1}{2}\ln \det K_{\bvec{\alpha}}(\bvec{x}) - \frac{N}{2}\ln 2\,\pi
\end{eqnarray}
where $\bvec{y}$ is the observed light curve, $f_{\bvec{\theta}}(\bvec{x})$ is
the transit model (parameterized by $\bvec{\theta}$), $\bvec{x}$ are the
\emph{predictive features} (more on these later).


\section{Discussion}

All of the code used in this project is available from
\url{http://github.com/dfm/turnstile} under the MIT open-source software
license.
This code (plus some dependencies) can be run to re-generate all of the
figures and results in this \paper; this version of the paper was generated
with git commit \texttt{\githash} (\gitdate).

\acknowledgments
It is a pleasure to thank
\ldots\
for helpful contributions to the ideas and code presented here.
This project was partially supported by the NSF (grant AST-0908357), NASA
(grant NNX08AJ48G), and the Moore--Sloan Data Science Environment at NYU.
This research made use of the NASA \project{Astrophysics Data System}.

\newcommand{\arxiv}[1]{\href{http://arxiv.org/abs/#1}{arXiv:#1}}
\begin{thebibliography}{}\raggedright

\bibitem[Ambikasaran \etal(2014)]{dfm-gp}
Ambikasaran, S., Foreman-Mackey, D., Greengard, L., Hogg, D.~W.,
\& O'Neil, M.\ 2014, \arxiv{1403.6015}

\bibitem[Carter \& Winn(2009)]{carter}
Carter, J.~A., \& Winn, J.~N.\ 2009, \apj, 704, 51 (\arxiv{0909.0747})

\bibitem[Gibson \etal(2012)]{gibson-gp}
Gibson, N.~P., Aigrain, S., Roberts, S., \etal\ 2012, \mnras, 419, 2683
(\arxiv{1109.3251})

\bibitem[Petigura \etal(2013a)]{petigura-a}
Petigura, E.~A., Marcy, G.~W., \& Howard, A.~W.\ 2013a, \apj, 770, 69
(\arxiv{1304.0460})

\bibitem[Petigura \etal(2013b)]{petigura}
Petigura, E.~A., Howard, A.~W., \& Marcy, G.~W.\ 2013b,
Proceedings of the National Academy of Science, 110, 19273 (\arxiv{1311.6806})

\bibitem[Rasmussen \& Williams(2006)]{gp}
Rasmussen, C.~E. \& Williams, C.~K.~I.\ 2006
Gaussian Processes for Machine Learning, MIT Press
(\href{http://www.gaussianprocess.org/gpml/}{online})

\bibitem[Roberts \etal(2013)]{roberts}
Roberts, S., McQuillan, A., Reece, S., \& Aigrain, S.\ 2013, \mnras, 435, 3639
(\arxiv{1308.3644})

\end{thebibliography}

\clearpage

% \begin{figure}[p]
% \begin{center}
% \includegraphics[width=\textwidth]{path/to/figure.pdf}
% \end{center}
% \caption{%
% A CAPTION.
% \figlabel{the-figure-label}}
% \end{figure}

\end{document}
